Given a periodic point \( \alpha \) of \( g \), the **period** of \( \alpha \) is the positive integer 
\[ n:= \mathrm{min} \{ k \ge 1 : g^k (\alpha) = \alpha \}.\]


We will let \( \mathrm{Per}_{ n } (g)  \) denote the <b class="def"> set of all \( n \)-periodic points</b>: 
\[ \mathrm{Per}_{ n } (g) := \{ \alpha \in \bar{\mathbb{F}}  : \text{\(\alpha\) has period \(n\) for \(g\)}\}. \]

Finally, given any subset \( S \subseteq  \bar{\mathbb{F}}  \), we will let 

denote the <b class="def"> set of all \( n \)-periodic points in \( S \)  </b>. 

Now fix any integer \( n \ge 1 \). Our key tool for studying the \( n \)-periodic points of \( f(x) \) is the <b class="def"> \( n \)-th dynatomic polynomial </b>:
\[ 
 	 \Phi_{n  }  \left( x \right) := \prod_{d \mid n} \left( f^d(x) - x \right)^{\mu(n/d)}
\]
where \( \mu : \mathbb{Z}  _{>0} \to \{0 , \pm 1\} \) is the <b class="def"> Mobius function </b>  given by $\mu(1) := 1$ and 
	\[
		\mu \left( p_1^{e_1} \ldots p_r^{e_r} \right) :=
		\left\{ 
		 		\begin{array}{lc} 
		 			0, & \text{if $e_j \ge 2$ for some $j \in [r]$;} \\[0.5em]
		 			(-1)^r, & \text{else.} 
		 		\end{array}   
		 \right.  
	\]

The utility of \(  \Phi_{n  }  \) is that its zeros are precisely the \( n \)-periodic points of \( f \). Indeed, let \( L_n \) denote the <b class="def"> splitting field </b> of \(  \Phi_{n  }     \), and let \( Z_n \) denote the . Then we have the following: 

 A consequence of the third and fourth statements above is that if we define \( r= \frac{\deg  \Phi_{n  }  \left( x \right)}{n} \), then we can partition \( Z_n \) as 
 \[
  Z = \bigsqcup_{i=1}^{r} A_i
 \]
 where \( A_i = \{\alpha_i , f(\alpha_i) , \ldots , f^{n-1} (\alpha_i) \} \)  for some \( \alpha_i \in L_n \)  for every \( i = 1, \ldots , r \) .

\[ 
  f(x) \mid_{t = c}  = (x^2 +t)|_{t = c} = x^2 +c = f_c (x)
\] 

for \( c \in \mathbb{Q} \),  
\[
  \Phi_{n,c} (x)  :=  \prod_{d \mid n} \left( f_c^d (x) - x \right) ^{\mu(n/d)} \in \mathbb{Q} [x]
\]

Notice that 
\[ 
	 \Phi_{n ,c }  \left( x \right)     =  \Phi_{n  }  \left( x \right)    |_{t = c}
\]  
so that the \( n \)-th dynatomic polynomial (specialized) at \( c \) deserves its name.

Let \( L_{n,c} / \mathbb{Q}  \) denote the <b class="def"> splitting field of \(  \Phi_{n ,c }  \)</b>, and let
\[ 
  Z_{n,c}:= \{ \alpha \in L_{n,c} :  \Phi_{n ,c }  \left( \alpha \right)    = 0\} 
\] 
denote the <b class="def"> (geometric) zero set of  \Phi_{n ,c } (x)    </b>.

In this specialized setting, we still have the containment \( \mathrm{Per}_{ n } (f_c)  \subseteq  Z_{n,c} \), but equality need not hold. (Recall that we saw that \( Z_n = \mathrm{Per}_{ n } (f)   \)).

Now, each of the field extensions \( L_n / K \) and \( L_{n,c} / \mathbb{Q}  \) is Galois, so we can discuss their Galois groups:  
\[ 
  G_n:= \mathrm{Gal}(L_n/K)   
\] 
is the <b class="def"> \( n \)-th dynatomic Galois group </b>, and 
\[ 
   G_{n,c}:= \mathrm{Gal}(L_{n,c} /\mathbb{Q} )   
\]
is the <b class="def"> \( n \)-th dynatomic Galois group (specialized) at \( c \)  </b>. 
\[ 
  G_n \cong \left( \mathbb{Z} / n \right)^{ r } \rtimes S_{ r }    
\] 
where \( S_r \) acts by automorphisms on \( \left( \mathbb{Z} / n \right)^{ r }   \) by permuting factors according to the natural action of \( S_r \) on \( [r]:=\{1 , \ldots, r\} \).  

This nice result begs the question: do we have a similarly nice description of the specialized dynatomic Galois groups \( G_{n,c}  \)?

If we define
\[ 
  D_n := \{ c \in \mathbb{Q} :  \text{\( \Phi_{n ,c } (x) \) is inseparable}   \},
\]
then we can give a partial answer to this natural question: 

If \( c \in \mathbb{Q} -D_n \), then \( G_{n,c} \le G_n \).  

By definition, the set \( D_n \) is the zero set of the discriminant of \(  \Phi_{n  }  \left( x \right) \), and from this it follows that \( D_n \) is a finite set. 

Knowing that \( G_{n,c} \le G_n \) may not be super helpful by itself. Indeed, the size \(  \left| G_n \right|   \) grows (roughly) exponentially with \( n \)---see table below---, so that the size of the subgroup lattice of \( G_n \) becomes incomprehensibly large very quickly.

Define
\[ 
  E_n := \{ c \in \mathbb{Q} - D_n : G_{n,c} \lneq G_n  \} 
\]
to be the set of all values \( c \in \mathbb{Q} -D_n \) for which \( G_{n,c}  \) embeds as a proper subgroup of \( G_n \). We call \( E_n \) the <b class="def"> \( n \)-th exceptional set </b>, and elements of \( E_n \) are called <b class="def"> exceptional values </b>. We are finally equipped to make a connection between dynatomic Galois groups and periodic points in \( \mathbb{Q}  \) for quadratic polynomials: 

Let \( c \in \mathbb{Q} -D_n \). If \( \mathrm{Per}_{ n } (f_c; \mathbb{ Q } )   \) is non-empty, then \( c \in E_n \). 

Recall from earlier that \(  \mathrm{F.P.}(n)   \) is a short hand for the statement: ``there are at most finitely many \( c \in \mathbb{Q}  \)  for which \( \mathrm{Per}_{ n } (f_c; \mathbb{ Q } )   \) is non-empty.'' 

\[ 
    \mathrm{Per}_{ n } (g) := \{ \alpha \in \bar{\mathbb{F}}  : \text{\( \alpha \) is \( n \)-periodic for \( g \)} \}   
\]

\(  \mathrm{F.P.}(n)   \), it is enough to show that \( E_n \) is finite.

Fix an integer \( n \ge 1 \). If \( E_n \) is a finite set, then \(  \mathrm{F.P.}(n)   \) is true.  

\(  \mathrm{F.P.}(n)   \) 

Fix an integer \( n \ge 1 \), and let \( H \) be any subgroup of our dynatomic Galois group \( G_n \). We will let \( L_n^{H} \) denote the <b class="def"> fixed field of \( H \) in \( L_n \)  </b>. Then each of the fields \( K \), \( L_n^{H}  \), and \( L_n \) is a <b class="def"> function field over \( \mathbb{Q}  \)  </b>. That is, for each \( \mathbb{F}  \in \{K, L_n^{H} , L_n\} \), the field \( F \) is a finitely generated extension of \( \mathbb{Q}  \) with transcendence degree \( 1 \) over     

the extension \( \mathbb{F}  / \mathbb{Q}  \) has transendence degree \( 1 \);  

\( \mathbb{Q}  \) is algebraically closed in \( \mathbb{F}   \). 

there are curves \( X \), \( Y \), and \( W \) corresponding to \( L_n \), \( L_n^{H} \), and \( K \) respectively. 

The curve \( W \) corresponding to \( K = \mathbb{Q} (t)  \) is \( \mathbb{P}  _{\mathbb{Q} } ^{1} \), the <b class="def"> projective line defined over \( \mathbb{Q}  \)  </b>  

The curve \( Y \) corresponding to \( L_n^{H}  \) is the <b class="def"> quotient curve </b> \( X_H \equiv X / H \) whose points are \( H \)-orbits under the action of \( H \) on \( X \). 

Now, if \( C \) is any curve defined over \( \mathbb{Q}  \), we will let \( C(\mathbb{Q} ) \) denote the <b class="def"> set of all \( \mathbb{Q}  \)-rational points of \( C \)  </b>.  

Then \( \mathbb{Q}  \)-rational points on dynatomic modular curves are closely related to exceptional values via the 

Let \( c \in \mathbb{Q} -D_n \). Then \( c \in E_n \) if and only if \( c \in \pi_H (X_H (\mathbb{Q}  )) \) for some proper subgroup \( H \lneq G_n \). 

\( X_H (\mathbb{Q} ) \) is finite for every proper subgroup \( H \) of G_n if and only if \( X_M (\mathbb{Q} ) \) is finite for every maximal subgroup \( M \) of G_n

Let \( C \) be any curve defined over \( \mathbb{Q}  \). The standard tool for showing that \( C(\mathbb{Q} ) \) is finite is Faltings' Theorem. Let \( g(C) \) denote the <b class="def"> genus of \( C \)  </b>.  

Our strategy then will be to show that for every proper subgroup \( H \lneq G_n \) we have that the genus \( g(X_H) \) of the dynatomic modular curve \( X_H \) is at least \( 2 \). 

By a standard reduction, we can do even better: it suffices to show that \( g(X_M) \ge 2 \) for every maximal subgroup \( M \lneq G_n \).  

Let \( H \) be any subgroup of \( G_n \). Then
\[ 
	2g(X_H) - 2 = (-2)[G_n:H] + \sum_{P \in \mathbb{P}  } \sum_{Q \mid P}  (e (Q \mid P)    -1),
\]
where the first sum runs over all places \( P \) of \( K \), the second sum runs over all places \( Q \) of \( L_n^H \) lying above \( P \), and \(  e (Q \mid P)     \) is the <b class="def"> ramification index </b>  of \( Q \mid P \).

 <div class="container">
        <div class="thm">
          <b class="def">
            Theorem:
          </b> 
          <p style="text-indent: 0">
            Let \( H \) be any subgroup of \( G_n \). Then
      \[ 
        2g(X_H) - 2 = (-2)[G_n:H] + \sum_{P \in \mathbb{P}  } \sum_{Q \mid P}  (e (Q \mid P)    -1),
      \]
      where the first sum runs over all places \( P \) of \( K \), the second sum runs over all places \( Q \) of \( L_n^H \) lying above \( P \), and \(  e (Q \mid P)     \) is the <b class="def"> ramification index </b>  of \( Q \mid P \).
          </p>
        </div> <!-- end div.thm -->
      </div> <!-- end div.container -->

      \( [{ }^{ g } I : { }^{ g } I \cap M ] \) 